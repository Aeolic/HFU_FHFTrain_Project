\chapter{Einleitung}

Im Gegensatz zu den meisten anderen Semesterprojekten stand bei unserem Projekt nicht im Vorraus was gemacht werden sollte und was das Ziel war, diese Entscheidung mussten wir selber fällen.\\
Dementsprechend startete das Semesterprojekt denkbar undefiniert. Schon in der Projektbeschreibung war angegeben, dass es viele Möglichkeiten gibt mit der vorhandenen Soft- und Hardware ein Semesterprojekt aufzustellen. Nach einer allgemeinen Einführung mussten wir uns also überlegen, was wir dieses Semester erarbeiten wollen. Wir wussten also ungefähr zu was der Zug in der Lage war und hatten jetzt die Möglichkeit dies zu erweitern.\\
Schnell kam die Überlegung auf, ob wir den Zug nicht mithilfe eines externen Gerätes steuern könnten, der die Position des Zugs kennt und mithilfe eines Steuer-Algorithmus den optimalen Weg für den Zug festlegen kann. Hier hofften wir viel Gelerntes anwenden zu können und ein Spannendes und auch Anspruchsvolles Projekt geschaffen zu haben. So legten wir uns also darauf fest, eine externe Zugsteuerung zu entwerfen.\\
Man kann sich jetzt fragen, was der optimale Weg für einen Zug sei. Wir haben festgelegt, dass es sich um einen Personenzug handeln soll, welcher möglichst effektiv Personen von bestimmten Startbahnhöfen zum jeweiligen Zielbahnhof einer Person bringen muss. Dabei soll der Zug immer den bestmöglichen Weg wählen und auch nur anhalten, wenn wirklich Personen ein- oder aussteigen wollen.\\
Durch das Benutzen eines externen Gerätes entstanden natürlich weitere Anforderungen, die wir beachten mussten:
\begin{itemize}
	\item Wie können die beiden Geräte miteinander kommunizieren?
	\item Wie können wir mit IBM Rhapsody, der Entwicklungsumgebung, welche wir benutzten, sowohl für den Zug als auch für die externe Steuereinheit kompilieren?
	\item Was muss der Zug genau können und was die externe Steuereinheit?
\end{itemize}
Ein weiterer wichtiger Punkt war, dass wir kein jungfräuliches Projekt vor uns hatten, sondern unser Projekt auf Vorgängerprojekten der letzten Semester basierte, wir also schon definierte Hardware hatten und dementsprechend auch dazugehörige Software.\\ 
In dieser Dokumentation werden, nebst einer allgemeinen Projektbeschreibung, die einzelnen Arbeitspakete aufgeführt, die in dem Projekt bearbeitet wurden.
