\chapter{Fazit}

Während des Semesterprojektes haben wir viel gelernt, besonders der Umgang mit IBM Rhapsody und die modellgetriebene Entwicklung war uns allen nicht vertraut. Auch die tiefen Einblicke in die Entwicklung für Embedded Systems und dabei benutzte Techniken und Werkzeuge waren uns neu.\\
Durch unsere wöchentlichen Projekttreffen wurde uns zudem ein Einblick in Projektarbeit gegeben, wie wir ihn noch nicht zuvor bekommen haben.\\
Zu beginn des Semesters war Rhapsody uns noch komplett unbekannt und jeder Schritt von uns dauerte sehr lange. Aufgrund dessen war uns auch der Nutzen von Rhapsody nicht klar. Wir haben vorher überwiegend nur mit Texteditoren oder IDEs wie Eclipse gearbeitet und die doch sehr komplexe Oberfläche von Rhapsody hat uns abgeschreckt. Gegen Ende des Semesters wurde uns dessen Mächtigkeit jedoch bewusst, als es zum Cross-Compiling der Software für verschiedene Verteilte Systeme kam. Nach dem Einrichten von Rhapsody hätte das Kompilieren kaum einfacher sein können. Auch die Kombination von Rhapsody mit der Versionsverwaltung \textit{git} hat uns am Anfang Kopfzerbrechen bereitet. Das automatische zusammenführen von gleichzeitig bearbeiteten Dateien war nicht möglich, was zu einer Aufteilung der Software in mehrere Pakete führte. Zudem wurden einige von Rhapsody erzeugten Dateien mithilfe einer \textit{.gitignore} von der Versionsverwaltung ausgeschlossen.\\
Wir hätten uns gewünscht am Ende des Semesters auch die Ziele der zweiten Version umgesetzt zu haben, aber aufgrund der langwierigen Einarbeit in die von uns verwendeten Tools und diverser Rückschläge im laufe des Semesters, darunter beispielsweise ein defekt des Zuges kurz vor Ende, war dies leider nicht möglich. Um diese zusätzlichen Ziele umzusetzen hätten wir jedoch auch bei perfektem Ablauf des Projektes mehr Zeit benötigt, wir haben den Aufwand zu Beginn also Unterschätzt.\\\\
Abschließend kann man sagen das wir sehr zufrieden mit dem Ausgang des Semesterprojektes sind und sher viele Erfahrungen auf unseren weiteren Weg mitnehmen.